
%----------------------------------------------------------------------------------------
%	PACKAGES AND OTHER DOCUMENT CONFIGURATIONS
%----------------------------------------------------------------------------------------

\documentclass[
	11pt, % Set the default font size, options include: 8pt, 9pt, 10pt, 11pt, 12pt, 14pt, 17pt, 20pt
	%t, % Uncomment to vertically align all slide content to the top of the slide, rather than the default centered
	%aspectratio=169, % Uncomment to set the aspect ratio to a 16:9 ratio which matches the aspect ratio of 1080p and 4K screens and projectors
]{beamer}


\usepackage{booktabs} % Allows the use of \toprule, \midrule and \bottomrule for better rules in tables

%----------------------------------------------------------------------------------------
%	SELECT LAYOUT THEME
%----------------------------------------------------------------------------------------

% Beamer comes with a number of default layout themes which change the colors and layouts of slides. Below is a list of all themes available, uncomment each in turn to see what they look like.

%\usetheme{default}
%\usetheme{AnnArbor}
%\usetheme{Antibes}
%\usetheme{Bergen}
%\usetheme{Berkeley}
%\usetheme{Berlin}
%\usetheme{Boadilla}
%\usetheme{CambridgeUS}
\usetheme{Copenhagen}
%\usetheme{Darmstadt}
%\usetheme{Dresden}
%\usetheme{Frankfurt}
%\usetheme{Goettingen}
%\usetheme{Hannover}
%\usetheme{Ilmenau}
%\usetheme{JuanLesPins}
%\usetheme{Luebeck}
% \usetheme{Madrid}
%\usetheme{Malmoe}
%\usetheme{Marburg}
%\usetheme{Montpellier}
%\usetheme{PaloAlto}
%\usetheme{Pittsburgh}
%\usetheme{Rochester}
%\usetheme{Singapore}
%\usetheme{Szeged}
%\usetheme{Warsaw}

%----------------------------------------------------------------------------------------
%	SELECT COLOR THEME
%----------------------------------------------------------------------------------------

% Beamer comes with a number of color themes that can be applied to any layout theme to change its colors. Uncomment each of these in turn to see how they change the colors of your selected layout theme.

%\usecolortheme{albatross}
% \usecolortheme{beaver}
%\usecolortheme{beetle}
%\usecolortheme{crane}
% \usecolortheme{dolphin}
% \usecolortheme{dove}
%\usecolortheme{fly}
% \usecolortheme{lily}
%\usecolortheme{monarca}
%\usecolortheme{seagull}
%\usecolortheme{seahorse}
% \usecolortheme{spruce}
% \usecolortheme{whale}
% \usecolortheme{wolverine}

%----------------------------------------------------------------------------------------
%	SELECT FONT THEME & FONTS
%----------------------------------------------------------------------------------------

% Beamer comes with several font themes to easily change the fonts used in various parts of the presentation. Review the comments beside each one to decide if you would like to use it. Note that additional options can be specified for several of these font themes, consult the beamer documentation for more information.

\usefonttheme{default} % Typeset using the default sans serif font
% \usefonttheme{serif} % Typeset using the default serif font (make sure a sans font isn't being set as the default font if you use this option!)
%\usefonttheme{structurebold} % Typeset important structure text (titles, headlines, footlines, sidebar, etc) in bold
%\usefonttheme{structureitalicserif} % Typeset important structure text (titles, headlines, footlines, sidebar, etc) in italic serif
%\usefonttheme{structuresmallcapsserif} % Typeset important structure text (titles, headlines, footlines, sidebar, etc) in small caps serif

%------------------------------------------------

% \usepackage{mathptmx} % Use the Times font for serif text
% \usepackage{palatino} % Use the Palatino font for serif text
\usepackage{fontawesome}
\usepackage{amsmath}
\usepackage{physics}
\usepackage{graphicx}
% \usepackage{helvet} % Use the Helvetica font for sans serif text
% \usepackage[default]{opensans} % Use the Open Sans font for sans serif text
% \usepackage[default]{FiraSans} % Use the Fira Sans font for sans serif text
% \usepackage[default]{lato} % Use the Lato font for sans serif text

%----------------------------------------------------------------------------------------
%	SELECT INNER THEME
%----------------------------------------------------------------------------------------

% Inner themes change the styling of internal slide elements, for example: bullet points, blocks, bibliography entries, title pages, theorems, etc. Uncomment each theme in turn to see what changes it makes to your presentation.

% \useinnertheme{default}
\useinnertheme{circles}
%\useinnertheme{rectangles}
% \useinnertheme{rounded}
% \useinnertheme{inmargin}

%----------------------------------------------------------------------------------------
%	SELECT OUTER THEME
%----------------------------------------------------------------------------------------

% Outer themes change the overall layout of slides, such as: header and footer lines, sidebars and slide titles. Uncomment each theme in turn to see what changes it makes to your presentation.

% \useoutertheme{default}
\useoutertheme{infolines}
% \useoutertheme{miniframes}
% \useoutertheme{smoothbars}
% \useoutertheme{sidebar}
% \useoutertheme{split}
% \useoutertheme{shadow}
% \useoutertheme{tree}
% \useoutertheme{smoothtree}

% \setbeamertemplate{footline} % Uncomment this line to remove the footer line in all slides
%\setbeamertemplate{footline}[page number] % Uncomment this line to replace the footer line in all slides with a simple slide count

% \setbeamertemplate{navigation symbols}{} % Uncomment this line to remove the navigation symbols from the bottom of all slides

% manual override to display all the content on the footer

% \setbeamertemplate{footline}{%
%     \leavevmode%
%     \hbox{%
%         \begin{beamercolorbox}[wd=.333333\paperwidth,ht=2.25ex,dp=1ex,center]{author in head/foot}%
%             \usebeamerfont{author in head/foot}\insertshortauthor
%         \end{beamercolorbox}%
%         \begin{beamercolorbox}[wd=.333333\paperwidth,ht=2.25ex,dp=1ex,center]{title in head/foot}%
%             \usebeamerfont{title in head/foot}\insertshorttitle
%         \end{beamercolorbox}%
%         \begin{beamercolorbox}[wd=.333333\paperwidth,ht=2.25ex,dp=1ex,right]{date in head/foot}%
%             \usebeamerfont{date in head/foot}\insertshortinstitute\hspace*{2em}
%             \insertframenumber{} / \inserttotalframenumber\hspace*{2ex} 
%         \end{beamercolorbox}}%
%     \vskip0pt%
% }


%----------------------------------------------------------------------------------------
%	PRESENTATION INFORMATION
%----------------------------------------------------------------------------------------

\title[Wobbling Motion]{Parity Partner Bands and the Wobbling Motion in 163Lu} % The short title in the optional parameter appears at the bottom of every slide, the full title in the main parameter is only on the title page

% \subtitle{Optional Subtitle} % Presentation subtitle, remove this command if a subtitle isn't required

\author[Robert Poenaru]{Robert Poenaru\texorpdfstring{$^{1,2}$}{(1,2)}} % Presenter name(s), the optional parameter can contain a shortened version to appear on the bottom of every slide, while the main parameter will appear on the title slide

\institute[IFIN-HH]{$^1$Doctoral School of Physics, Univ. of Bucharest \\ $^2$IFIN-HH, Magurele \\ \smallskip \textit{robert.poenaru@drd.unibuc.ro}} % Your institution, the optional parameter can be used for the institution shorthand and will appear on the bottom of every slide after author names, while the required parameter is used on the title slide and can include your email address or additional information on separate lines

\date[\today]{16th International Conference on Nuclear Structure Properties (NSP2023) \\ \today} % Presentation date or conference/meeting name, the optional parameter can contain a shortened version to appear on the bottom of every slide, while the required parameter value is output to the title slide

%----------------------------------------------------------------------------------------

\begin{document}

%----------------------------------------------------------------------------------------
%	TITLE SLIDE
%----------------------------------------------------------------------------------------

{
\setbeamertemplate{headline}{}
\begin{frame}
	\titlepage % Output the title slide, automatically created using the text entered in the PRESENTATION INFORMATION block above
	% \vspace{\baselineskip}
\end{frame}
}

%----------------------------------------------------------------------------------------
%	TABLE OF CONTENTS SLIDE
%----------------------------------------------------------------------------------------

% The table of contents outputs the sections and subsections that appear in your presentation, specified with the standard \section and \subsection commands. You may either display all sections and subsections on one slide with \tableofcontents, or display each section at a time on subsequent slides with \tableofcontents[pausesections]. The latter is useful if you want to step through each section and mention what you will discuss.

\begin{frame}
	\frametitle{Presentation Overview} % Slide title, remove this command for no title
	
	\tableofcontents % Output the table of contents (all sections on one slide)
	%\tableofcontents[pausesections] % Output the table of contents (break sections up across separate slides)
\end{frame}

%----------------------------------------------------------------------------------------
%	PRESENTATION BODY SLIDES
%----------------------------------------------------------------------------------------

\section{Aim and Motivation}

\begin{frame}
	\frametitle{Motivation}
	\begin{block}{Motivation}
		\textbf{Nuclear Triaxiality} has become a \emph{hot topic} within the scientific community.
		\begin{itemize}
			\item Identifying nuclei with triaxial deformations represents a real \textbf{experimental} and \textbf{theoretical} challenge
			\item Experimental side: large setups, complex electronics, 
			\item Theoretical side: cumbersome models, approximations, abstractions...
		\end{itemize}
	\end{block}
\end{frame}


\begin{frame}
	\frametitle{Nuclear facilities}
	\begin{columns}
		\column{0.4\textwidth}
		\begin{figure}
		\centering
		\includegraphics[width=0.86\textwidth]{figures/gsfig.jpg}
		\caption{Gammasphere detector, ANL-ATLAS USA. \textit{Source: aps.org}}
	\end{figure}
	\column{0.6\textwidth}
	\begin{figure}
		\centering
		\includegraphics[width=0.98\textwidth]{figures/isolde_cern_2.png}
		\caption{a) IDS detector, CERN. \textit{Source: isolde.web.cern.ch} b) JUROGAM II, Finland. \textit{Source: twitter.com}}
		\end{figure}
	\end{columns}
\end{frame}

\begin{frame}
	\frametitle{Fingerprints of Triaxiality}
	\begin{block}{Evidence \faSearch}
		\begin{itemize}
			\item Currently, there are \textbf{only two} well-established phenomena uniquely attributed to triaxial deformation.
			\begin{enumerate}
				\item Wobbling Motion WM (\emph{Bohr and Mottelson, 1950s})
				\item Chiral Motion $\chi$M (\emph{Frauendorf, 1997})
			\end{enumerate}
			\item These two can be measured/detected experimentally.
		\end{itemize}
	\end{block}
	\begin{alertblock}{Experimental observations \faSearch}
		First experimental evidence for \textbf{nuclear wobbling motion} in 2001.
	\end{alertblock}
	\begin{exampleblock}{\textbf{Goal} \faClipboard}
		\textbf{Describe the elusive character of nuclear triaxiality through theoretical models on Wobbling Motion.}
	\end{exampleblock}
\end{frame}

\begin{frame}
	\frametitle{Theoretical Models}
	\begin{block}{Theoretical Models for the study of triaxial nuclei}
		\begin{enumerate}
			\item \emph{Harmonic Approximation} (Bohr and Mottelson, 1975)
			\item Triaxial-Rotor-Model (Davydov and Filippov, 1958)
			\item Boson-approximations (Tanabe, 1971)
			\item Particle Rotor Model (Hamamoto, 2002)
			% \item Tilted-axis wobbling (Frauendorf and Meng, 1997)
			\item Collective Hamiltonians (Chen, 2014)
		\end{enumerate}
		\emph{Others: RPA, Mean-Field Theories, GCM+AMP...}
	\end{block}
	
	\begin{exampleblock}{More recent work on wobbling motion..}
		\begin{itemize}
			\item RPA for odd-mass nuclei, Raduta et al (PRC, 2017)
			\item Tilted-axis wobbling, Budaca (PRC, 2018)
			\item Tilted-Precession (TiP), Lawrie et. al. (PRC, 2020)
			\item PRM+HA for $^{163}$Lu, R. Poenaru (IJMPE, 2021)
		\end{itemize}
	\end{exampleblock}

\end{frame}

		
\section{Nuclear Deformation}

\begin{frame}
	\frametitle{Nuclear Shapes (in the context of WM)}
	\begin{exampleblock}{Nuclear Radius}
		The \textbf{shape} of the nucleus is most generally described in terms of the \emph{nuclear radius}:
		\begin{align}
			R(\theta,\varphi;t)=R_0\left(1+\sum_{\lambda=0}^{^\infty}\sum_{\mu=-\lambda}^\lambda\alpha_{\lambda\mu}(t)Y_\lambda^\mu(\theta,\varphi)\right)
		\end{align}
	\end{exampleblock}
	% \begin{itemize}
	% 	\item The $\alpha_{\lambda\mu}$ are collective coordinates $\Longrightarrow$ \emph{vibrations of the nucleus}.
	% 	\item $Y_\lambda^\mu$ are the spherical harmonics.
	% \end{itemize}
	\begin{block}{Quadrupole radius - pure quadrupole deformations}

		\begin{itemize}
			\item Most relevant modes are the \textbf{quadrupole vibrations} $\lambda=2$ $\Longrightarrow$ \emph{Play a crucial role in the rotational spectra of nuclei}
		\end{itemize}
		\begin{align}
			R(\theta,\varphi)=R_0\left(1+\sum_\mu\alpha_{2\mu}Y_2^\mu(\theta,\varphi)\right)\ ,
		\end{align}
	\end{block}
\end{frame}

\begin{frame}
	\frametitle{Nuclear Shapes II}
	\begin{block}{Collective coordinates}
		\begin{itemize}
			\item Most of the nuclei are either \textbf{spherical} or \textbf{axially symmetric} in their ground-state.
		\end{itemize}
	\end{block}
	\begin{figure}
		\centering
		\includegraphics[scale=0.38]{figures/nuclear_shapes.png}
		\caption{\textbf{spherical:} $\beta_2=0$\ \textbf{prolate:} $\beta_2>0$\ \textbf{oblate:} $\beta_2<0$}
	\end{figure}
\end{frame}

\begin{frame}
	\frametitle{Nuclear Shapes III}
	\vspace{-0.2cm}
	\begin{alertblock}{Non-axial shapes}
		\begin{itemize}
			\item Coordinates $\alpha_{2\mu}$ can be reduced to only two \emph{deformation parameters}: $\beta_2$ (\emph{eccentricity}) and $\gamma$ (\textbf{triaxiality}).
			\item The triaxiality parameter $\gamma$ (\textit{Bohr, 1969}): departure from axial symmetry
		\end{itemize}
	\end{alertblock}
	\vspace{-0.2cm}
	\begin{figure}
		\centering
		\includegraphics[scale=0.42]{figures/nice_diagram.pdf}
		\includegraphics[scale=0.19]{figures/triaxial-shape.pdf}
		\vspace{-0.41cm}
		\caption{Source: \emph{R. Poenaru, PhD Thesis (WIP)}}
	\end{figure}
\end{frame}


\section{Wobbling Motion}

\begin{frame}
	\frametitle{Wobbling Motion}
		\begin{columns}
			\begin{column}{0.6\textwidth}
				\begin{block}{Characteristics}
					\begin{itemize}
						\item MOI anisotropy $\rightarrow$ the \emph{main rotation} around $\mathcal{J}_\text{max}$ is disturbed by the other two axes  $\rightarrow$ \textbf{resulting motion of the rotating nucleus has an oscillating behavior}
						\item The \textbf{total angular momentum } (a.m.) \textbf{precesses} and \textbf{wobbles} around $\mathcal{J}_\text{max}$
						\item Tilting by an energy quanta $\sim$ \emph{vibrational character} $\rightarrow$ \textbf{wobbling phonon} $n_w=0,1,2...$
						% \item Predicted: more than 50 years ago. First confirmed for $^{163}$Lu (\textit{Ødegård} et. al., 2001)
						\item Currently confirmed wobblers $A\approx[100,130,160,180]$.
					\end{itemize}
				\end{block}
			\end{column}
			\begin{column}{0.4\textwidth}
				\begin{figure}
					\includegraphics[width=\textwidth]{figures/wobbling-schematic.png}
					\vspace{-0.7cm}
					\caption{Geometrical interpretation of a wobbling triaxial nucleus: \emph{rotation + precession + oscillation}. \emph{R. Poenaru, PhD Thesis (WIP)}}
				\end{figure}
			\end{column}
		\end{columns}
\end{frame}

\begin{frame}
	\frametitle{Simple wobbler - Spectra}
	\vspace{-0.2cm}
	\begin{figure}
		\centering
		\includegraphics[scale=0.33]{figures/precessional_cone_2.pdf}
		\includegraphics[scale=0.28]{figures/wobbling_n_schematic-2.pdf}
		\includegraphics[scale=0.18]{figures/triaxial-shapes-even-A.pdf}
		% \caption{The precessional cone of the total angular momentum $\rightarrow$ wobbling excitations $\rightarrow$ tilting of $\mathbf{I}$.}
	\end{figure}
	\vspace{-0.5cm}
	\begin{figure}
		\centering
		\includegraphics[scale=0.35]{figures/wobbling_n_schematic-1.pdf}
		\includegraphics[scale=0.35]{figures/wobblingFreq-evenA.pdf}
		\vspace{-0.3cm}
		\caption{precession of $\mathbf{I}$ $\rightarrow$ wobbling excitations $\rightarrow$ tilting of $\mathbf{I}$ $\rightarrow$ tilting of $\mathbf{I}$. \emph{R. Poenaru, PhD Thesis (WIP)}}
	\end{figure}
\end{frame}

\section{Wobbling motion in odd-mass nuclei}

\begin{frame}
	\frametitle{WM in odd-mass nuclei}

	\begin{exampleblock}{Triaxial Rotor + Particle Model (PRM)}
		\begin{itemize}
			\item The WM in odd-mass nuclei can be described through the \textbf{Particle - Rotor Model} (Hamamoto, 2002).
			\item System is described by the interaction between an even-even triaxial core + single nucleon
			\item Nucleon is moving in a \emph{quadrupole deformed mean field} generated by the core
		\end{itemize}
		\begin{align}
			H&={\color{red}H_\text{rot}}+{\color{blue}H_sp}=\nonumber\\
			&={\color{red}\sum_{k=1,2,3}A_k\left(\hat{I}_k-\hat{j}_k\right)^2}+\nonumber\\
			&+{\color{blue}\frac{V}{j(j+1)}\left[\left(3\hat{j}_3^2-\mathbf{j}^2\right)\cos\gamma-\sqrt{3}\left(\hat{j}_1^2-\hat{j}_2^2\right)\sin\gamma\right]+\epsilon_j}
		\end{align}
	\end{exampleblock}
\end{frame}

\subsection{Formalism applied to 163-Lu}

\begin{frame}
	\frametitle{WM in 163-Lu - old vs. new pictures}
		\begin{columns}
			\begin{column}{0.5\textwidth}
				\vspace{-0.3cm}
				\begin{block}{$^{163}$Lu - old picture}
					\begin{itemize}
						\item Accepted in literature (Hamamoto et. al. 2003, Raduta et. al. 2017).
						\item Nucleus has \textbf{four} known wobbling bands: $n_w=0,1,2,3$.
						\item TSD1-3: $\mathbf{R}^+$ + $j^+=\pi(i_{13/2}$). TSD4: $\mathbf{R}^+$ + $j^-=\pi(h_{9/2})$.
					\end{itemize}
				\end{block}
				\begin{figure}
					\vspace{-0.4cm}
					\includegraphics[scale=0.12]{figures/triaxial-shapes-oddA-2.png}
				\end{figure}
			\end{column}
			\vspace{0.5cm}
			\begin{column}{0.5\textwidth}
				\begin{alertblock}{$^{163}$Lu - new picture}
					\begin{itemize}
						\item All bands described by a \emph{unique single particle}: $j'^+=\pi(i_{13/2})$.
						\item TSD1: ($\mathbf{R}^+=0,2,4)+j'^+$. TSD1 and 2: \emph{signature partner bands} (Raduta et. al. 2020).
						\item TSD2: ($\mathbf{R}^+=1,3,5)+j'^+$.
						\item TSD3: one-phonon wobbling band ($n_w=1$), built on TSD2.
						\item TSD4: ($\mathbf{R}^-=1,3,5)+j'^+$. TSD2 and 4 are \textbf{parity partner bands} (Poenaru et. al. 2021).
					\end{itemize}
				\end{alertblock}
			\end{column}
		\end{columns}
\end{frame}

\begin{frame}
	\frametitle{Theoretical model - semiclassical approach}
	\vspace{-0.5cm}
	\begin{block}{Algorithm}
		\begin{itemize}
			\item Apply the \emph{variational principle} (VP) for the initial Hamiltonian using the trial function:
			\vspace{-0.5cm}
			\begin{align}
				\ket{\Psi_{Ij;M}}=\mathcal{N}{\color{red}e^{z\hat{I}_-}}{\color{blue}e^{s\hat{j}_-}}{\color{red}\ket{IMI}}\otimes{\color{blue}\ket{jj}}
			\end{align}
			\item $(z,\hat{I},\ket{IMI})$ - core states (for the even-even core)
			\item $(s,\hat{j},\ket{jj})$ - single-particle states (for the proton)
		\end{itemize}
	\end{block}
	\vspace{-0.2cm}
	\begin{exampleblock}{Equations of motion}
		The VP principle will lead to a system of \emph{equations of motion}:
		\begin{align}
			{\color{red}\frac{\partial\mathcal{H}}{\partial r}=\dot{\varphi}\ ,\ \frac{\partial\mathcal{H}}{\partial \varphi}=-\dot{r}}\ ,\ {\color{blue}\frac{\partial\mathcal{H}}{\partial t}=\dot{\psi}\ ,\ \frac{\partial\mathcal{H}}{\partial \psi}=-\dot{t}}\ .
		\end{align}
		\textbf{Solve:} $\Omega^4+B\Omega^2+C=0 \longrightarrow$ obtain two $\Omega\equiv$ wobbling frequencies ${\color{red}\Omega_1}$ (core) and ${\color{blue}\Omega_2}$ (odd-particle).
	\end{exampleblock}
\end{frame}

\begin{frame}
	\frametitle{Excitation energies}
	\begin{block}{Energy scheme}
		\begin{itemize}
			\item From the equations of motion and the two solutions for $\Omega$, the spectrum becomes:
			\begin{align}
				E_I^\text{TSD1}=&\epsilon_{j}+\mathcal{H}_\text{min}^{(I,j)}+\mathcal{F}_{00}^{I},\ I=R+j,\ R=0,2,4,\dots\nonumber\\
				E_I^\text{TSD2}=&\epsilon_{j_1}+\mathcal{H}_\text{min}^{(I,j)}+\mathcal{F}_{00}^{I},\ I=R+j,\ R=0,2,4,\dots\nonumber\\
				E_I^\text{TSD3}=&\epsilon_{j}+\mathcal{H}_\text{min}^{(I-1,j)}+\mathcal{F}_{10}^{I-1},\ \text{one-phonon wobbling band}\nonumber\\
				E_I^\text{TSD4}=&\epsilon_{j_2}+\mathcal{H}_\text{min}^{(I,j)}+\mathcal{F}_{00}^{I},\ I=R+j,\ R=0,2,4,\dots
			\end{align}
			\item Phonon term (contains the wobbling frequencies $\Omega$, R. Poenaru, IJMPE 2021):
			\begin{align}
				\mathcal{F}_{n_{w_1}n_{w_2}}^{I}=\Omega_1^I\left(n_{w_1}+\frac{1}{2}\right)+\Omega_2^I\left(n_{w_2}+\frac{1}{2}\right)
			\end{align}
		\end{itemize}
	\end{block}
\end{frame}

\begin{frame}
	\frametitle{Schematic representation of the new picture}
	\vspace{-0.3cm}
	\begin{figure}
		\centering
		\includegraphics[scale=0.45]{figures/W2_Approach.pdf}
	\end{figure}
\end{frame}
	
\subsection{Results}

\begin{frame}
	\frametitle{Results}
	\begin{exampleblock}{Fitting method}
		Apply fitting on a model with four free parameters: 3 moments of inertia $\mathcal{I}_{1,2,3}$, and the single-particle potential strength $V$.
		\vspace{-0.2cm}
		\begin{align}
			\chi^2=\frac{1}{N_T}\sum_i\frac{\left(E_\text{exp}^{(i)}-E_\text{th}^{(i)}\right)}{E_\text{exp}^{(i)}}
		\end{align}
		\vspace{-1cm}
		\begin{table}
			\caption{The obtained parameter set:}
			\centering
				\begin{tabular}{lllll}
					\hline
					$\mathcal{I}_1$ $\left[\hbar^2/\text{MeV}\right]$ & $\mathcal{I}_2$ $\left[\hbar^2/\text{MeV}\right]$ & $\mathcal{I}_3$ $\left[\hbar^2/\text{MeV}\right]$ & $\gamma$ [deg. ] & $V$ [\text{MeV}] \\
					\hline
					\hline
					72              & 15              & 7               & 22       & 2.1\\
					\hline
				\end{tabular}
			\end{table}
	\end{exampleblock}
	\textbf{Full work published in three papers: one in IJMPE 2021, two in RJP 2021.}
\end{frame}


\begin{frame}
	\frametitle{Excitation energies}
	\begin{center}
		\includegraphics[width=0.33\linewidth]{figures/DoubleShift_TSD1.pdf}
		\includegraphics[width=0.33\linewidth]{figures/DoubleShift_TSD1.pdf}
		\includegraphics[width=0.33\linewidth]{figures/DoubleShift_TSD1.pdf}
		\includegraphics[width=0.33\linewidth]{figures/DoubleShift_TSD1.pdf}
	\end{center}
\end{frame}

\begin{frame}
	\frametitle{Moments of Inertia}
	\begin{figure}	
		\begin{center}
			\includegraphics[width=0.49\textwidth]{figures/rigid-mois.png}
			\includegraphics[width=0.49\textwidth]{figures/hydro-mois.png}
			\caption{The rigid (\textbf{left}) and hydrodynamical (\textbf{right}) MOI as function of $\gamma$.}
		\end{center}
	\end{figure}
\end{frame}


\begin{frame}
	\frametitle{Conclusions}
	\begin{itemize}
		\item Wobbling motion has been described in the context of triaxial nuclei.
		\item The particle + rotor model was implemented in a \emph{new picture}, to describe the band structure of $^{163}$Lu.
		\begin{itemize}
			\item Variational principle applied to H
			\item semi-classical procedure to obtain equations of motion
			\item two wobbling frequencies emerge in the model
		\end{itemize}
		\item The four triaxial bands were numerically evaluated in terms of their excitation energies trough a fitting procedure.
		\item Moments of inertia show \emph{irrotational character} of the triaxial nucleus.
		\item Agreement with experimental data was obtained
	\end{itemize}
\end{frame}

%----------------------------------------------------------------------------------------
%	CLOSING SLIDE
%----------------------------------------------------------------------------------------

\begin{frame}[plain] % The optional argument 'plain' hides the headline and footline
	\begin{center}
		\bigskip\bigskip % Vertical whitespace
		{\Huge Thank you for your attention !}
	\end{center}
\end{frame}

%----------------------------------------------------------------------------------------

\end{document} 